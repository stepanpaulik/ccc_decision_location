% ARTICLE 2 ----
% This is just here so I know exactly what I'm looking at in Rstudio when messing with stuff.
% Options for packages loaded elsewhere
\PassOptionsToPackage{unicode}{hyperref}
\PassOptionsToPackage{hyphens}{url}
%
\documentclass[
  11pt,
]{article}
\usepackage{lmodern}
\usepackage{amssymb,amsmath}
\usepackage{ifxetex,ifluatex}
\ifnum 0\ifxetex 1\fi\ifluatex 1\fi=0 % if pdftex
  \usepackage[T1]{fontenc}
  \usepackage[utf8]{inputenc}
  \usepackage{textcomp} % provide euro and other symbols
\else % if luatex or xetex
  \usepackage{unicode-math}
  \defaultfontfeatures{Scale=MatchLowercase}
  \defaultfontfeatures[\rmfamily]{Ligatures=TeX,Scale=1}
\fi
% Use upquote if available, for straight quotes in verbatim environments
\IfFileExists{upquote.sty}{\usepackage{upquote}}{}
\IfFileExists{microtype.sty}{% use microtype if available
  \usepackage[]{microtype}
  \UseMicrotypeSet[protrusion]{basicmath} % disable protrusion for tt fonts
}{}
\makeatletter
\@ifundefined{KOMAClassName}{% if non-KOMA class
  \IfFileExists{parskip.sty}{%
    \usepackage{parskip}
  }{% else
    \setlength{\parindent}{0pt}
    \setlength{\parskip}{6pt plus 2pt minus 1pt}
    }
}{% if KOMA class
  \KOMAoptions{parskip=half}}
\makeatother
\usepackage{xcolor}
\IfFileExists{xurl.sty}{\usepackage{xurl}}{} % add URL line breaks if available
\urlstyle{same} % disable monospaced font for URLs
\usepackage[margin=1in]{geometry}
\setlength{\emergencystretch}{3em} % prevent overfull lines
\providecommand{\tightlist}{%
  \setlength{\itemsep}{0pt}\setlength{\parskip}{0pt}}
\setcounter{secnumdepth}{-\maxdimen} % remove section numbering

\ifluatex
  \usepackage{selnolig}  % disable illegal ligatures
\fi
\newlength{\cslhangindent}
\setlength{\cslhangindent}{1.5em}
\newlength{\csllabelwidth}
\setlength{\csllabelwidth}{3em}
\newenvironment{CSLReferences}[2] % #1 hanging-ident, #2 entry spacing
 {% don't indent paragraphs
  \setlength{\parindent}{0pt}
  % turn on hanging indent if param 1 is 1
  \ifodd #1 \everypar{\setlength{\hangindent}{\cslhangindent}}\ignorespaces\fi
  % set entry spacing
  \ifnum #2 > 0
  \setlength{\parskip}{#2\baselineskip}
  \fi
 }%
 {}
\usepackage{calc}
\newcommand{\CSLBlock}[1]{#1\hfill\break}
\newcommand{\CSLLeftMargin}[1]{\parbox[t]{\csllabelwidth}{#1}}
\newcommand{\CSLRightInline}[1]{\parbox[t]{\linewidth - \csllabelwidth}{#1}\break}
\newcommand{\CSLIndent}[1]{\hspace{\cslhangindent}#1}


\title{CYBERSPACE Abstract: Locating Czech Constitutional Court
Decisions in Doctrine Space and Measuring its Caselaw Consistency}
\author{true \and true}
\date{July 29, 2023}

% Jesus, okay, everything above this comment is default Pandoc LaTeX template. -----
% ----------------------------------------------------------------------------------
% I think I had assumed beamer and LaTex were somehow different templates.


\usepackage{kantlipsum}

\usepackage{abstract}
\renewcommand{\abstractname}{}    % clear the title
\renewcommand{\absnamepos}{empty} % originally center

\renewenvironment{abstract}
 {{%
    \setlength{\leftmargin}{0mm}
    \setlength{\rightmargin}{\leftmargin}%
  }%
  \relax}
 {\endlist}

\makeatletter
\def\@maketitle{%
  \newpage
%  \null
%  \vskip 2em%
%  \begin{center}%
  \let \footnote \thanks
      {\fontsize{18}{20}\selectfont\raggedright  \setlength{\parindent}{0pt} \@title \par}
    }
%\fi
\makeatother


\title{CYBERSPACE Abstract: Locating Czech Constitutional Court
Decisions in Doctrine Space and Measuring its Caselaw Consistency }

\date{}

\usepackage{titlesec}

% 
\titleformat*{\section}{\large\bfseries}
\titleformat*{\subsection}{\normalsize\itshape} % \small\uppercase
\titleformat*{\subsubsection}{\normalsize\itshape}
\titleformat*{\paragraph}{\normalsize\itshape}
\titleformat*{\subparagraph}{\normalsize\itshape}

% add some other packages ----------

% \usepackage{multicol}
% This should regulate where figures float
% See: https://tex.stackexchange.com/questions/2275/keeping-tables-figures-close-to-where-they-are-mentioned
\usepackage[section]{placeins}



\makeatletter
\@ifpackageloaded{hyperref}{}{%
\ifxetex
  \PassOptionsToPackage{hyphens}{url}\usepackage[setpagesize=false, % page size defined by xetex
              unicode=false, % unicode breaks when used with xetex
              xetex]{hyperref}
\else
  \PassOptionsToPackage{hyphens}{url}\usepackage[draft,unicode=true]{hyperref}
\fi
}

\@ifpackageloaded{color}{
    \PassOptionsToPackage{usenames,dvipsnames}{color}
}{%
    \usepackage[usenames,dvipsnames]{color}
}
\makeatother
\hypersetup{breaklinks=true,
            bookmarks=true,
            pdfauthor={Štěpán Paulík (Humboldt Universität zu Berlin,
Institut für Sozialwissenschaften,
\href{mailto:stepan.paulik.1@hu-berlin.de}{\nolinkurl{stepan.paulik.1@hu-berlin.de}}) and Jaromír
Fronc (Charles University, Faculty of Law,
\href{mailto:jaromir.fronc@gmail.com}{\nolinkurl{jaromir.fronc@gmail.com}})},
             pdfkeywords = {},
            pdftitle={CYBERSPACE Abstract: Locating Czech Constitutional
Court Decisions in Doctrine Space and Measuring its Caselaw
Consistency},
            colorlinks=true,
            citecolor=blue,
            urlcolor=blue,
            linkcolor=magenta,
            pdfborder={0 0 0}}
\urlstyle{same}  % don't use monospace font for urls

% Add an option for endnotes. -----



% This will better treat References as a section when using natbib
% https://tex.stackexchange.com/questions/49962/bibliography-title-fontsize-problem-with-bibtex-and-the-natbib-package

% set default figure placement to htbp
\makeatletter
\def\fps@figure{htbp}
\makeatother



\usepackage{longtable}
\LTcapwidth=.95\textwidth
\linespread{1.05}
\usepackage{hyperref}

\newtheorem{hypothesis}{Hypothesis}


% trick for moving figures to back of document
% really wish we'd knock this shit off with moving tables/figures to back of document
% but, alas...

% 
% Optional code chunks ------
% SOURCE: https://stackoverflow.com/questions/50702942/does-rmarkdown-allow-captions-and-references-for-code-chunks



\begin{document}

% \textsf{\textbf{This is sans-serif bold text.}}
% \textbf{\textsf{This is bold sans-serif text.}}


% \maketitle

{% \usefont{T1}{pnc}{m}{n}
\setlength{\parindent}{0pt}
\thispagestyle{plain}
{%\fontsize{18}{20}\selectfont\raggedright
\maketitle  % title \par

}




{
   \vskip 13.5pt\relax \normalsize\fontsize{11}{12}
   \MakeUppercase{Štěpán Paulík}, \small{Humboldt Universität zu Berlin,
Institut für Sozialwissenschaften,
\href{mailto:stepan.paulik.1@hu-berlin.de}{\nolinkurl{stepan.paulik.1@hu-berlin.de}}}   \par \vskip -3.5pt \MakeUppercase{Jaromír
Fronc}, \small{Charles University, Faculty of Law,
\href{mailto:jaromir.fronc@gmail.com}{\nolinkurl{jaromir.fronc@gmail.com}}}   

}

}








\begin{abstract}

%    \hbox{\vrule height .2pt width 39.14pc}

    \vskip 8.5pt % \small

\noindent \small{Our research explores the estimation of positions of
Czech Constitutional Court decisions in a doctrine space using Bayesian
statistical model. Traditional methods of estimating ideological
positions suffer from limitations, prompting the adoption of new
text-as-data approaches empowered by advances in computational
technology and statistics. Two research teams have attempted to overcome
previous constraints and estimate judicial positions more accurately,
one in the SCOTUS context and one in the German lower courts context.
Our study implements the method of Clark and Lauderdale of estimating
the locations of SCOTUS decisions with positive or negative references
to its caselaw: the closer decisions are to each other, the more likely
they are to cite themselves positively and vice-versa. We combine our
own dataset of all CCC decisions with the data on citations provided to
us by Beck. We use the programming language R and the Bayesian engine
Stan to estimate the positions employing the Bayesian model of Clark and
Lauderdal. Estimating the positions allows us to examine the consistency
of the CC's case law across different senates and the plenum. We narrow
our analysis to areas of law in common doctrine space that are prone to
inconsistency, namely restitution cases and costs of civil proceedings.
The research contributes to harnessing the potential of machine learning
and quantitative methods in legal research and clarifies the factors
influencing caselaw consistency.}


%    \hbox{\vrule height .2pt width 39.14pc}


\end{abstract}


\vskip -8.5pt


 % removetitleabstract


\setlength{\parindent}{16pt}
\setlength{\parskip}{0pt}

% We'll put doublespacing here
% Remember to cut it out later before bib
\vspace{30pt}

Especially American empirical legal science has long been trying to
estimate the ideological position of judges or judicial decisions,
especially in the context of SCOTUS (Supreme Court of the United
States). Based on this estimation, the research seeks to answer
subsequent sociological research questions. Spaeth-Segal (Segal et al.
1995) and Martin-Quinn (Martin and Quinn 2002) scores are among the
examples of attempts to quantify the ideological position of judges.
However, these traditional estimation methods suffer from various
shortcomings, such as measuring ideological positions indirectly (based
on the party the decision was made against) or using data not available
outside the American context (information about judges' voting) and,
thus, making the research near impossible to replicate elsewhere.

Recent developments in information technology enable more precise and
diverse work with data, specifically with text as data. This progress is
starting to make its way into social and legal sciences. Two research
teams have utilised the progress to overcome the aforementioned
limitations in traditional estimation efforts of judicial decisions and
judges' positions. Both teams operate on the premise that the ``closer''
judicial decisions or other legal documents are to each other, the
greater the likelihood that they will refer to one another. They employ
similar statistical models to estimate the position of a given decision
based on this premise. Moreover, their methods share the use of Bayesian
statistics, whose expansion has been facilitated by the development of
computational techniques. Despite these similarities, there is a
fundamental difference between the two research teams in determining
which intertextual reference is relevant for identifying the mutual
position of judicial decisions.

Clark and Lauderdale examine SCOTUS, which, like the Czech
Constitutional Court (Ústavní soud - ÚS), primarily cites its own
decisions (Clark and Lauderdale 2010). Therefore, they consider only
these citations, which they further divide into positive and negative
based on their relationship to the cited decision. In contrast, Arnold,
Engst, and Gschwend study German general courts of lower instances,
which primarily refer to various legal provisions and decisions of other
courts (mostly higher in the judicial pyramide) (Arnold, Engst, and
Gschwend 2023). Hence, they only use information about references to
other legal texts, which can include not only decisions of the same
court (as in the case of SCOTUS) but also provisions of any legal
regulation or decisions of any court. Despite these differences, the
outcome of both efforts is similar: estimating the placement of judicial
decisions on a common dimension, whose substantive meaning must be
subsequently interpreted by researchers.

The objective of our contribution is to present research in which we
apply Clark and Lauderdale's method to the Czech Constitutional Court
since we believe their approach is more suitable for the institutional
setup of the Constitutional Court, where decisions refer primarily to
its own case law, and references to legal provisions are not as diverse
(Constitution, Charter of Fundamental Rights and Freedoms). However, we
also intend to try the German method to compare the outcomes.

Based on the estimated positions of Constitutional Court decisions, we
will examine primarily the consistency of the Court's case law across
senates and the plenum. The interplay between the consistency of case
law and the Court's internal structure has already been the subject of
quantitative empirical research (Fjelstul 2021), which we build upon.

The case law of the Constitutional Court is inconsistent due to the
massive number of constitutional complaints and the conscious or
unconscious failure to unify the case law. The 3-member senates, into
which the 15-member Constitutional Court is divided, either are unaware
of conflicting case law, ignore it, or employ tricks to avoid its
binding nature. Moreover, it can be often difficult even to identify
that two decisions are in conflict with each other.

Scholarly literature provides examples of areas where the Constitutional
Court's decision-making practice is particularly inconsistent (e.g.,
restitution matters). It can be assumed that the level of inconsistency
may be related to the value-based or legal-political dimension of the
issue, which causes different judges to tend to decide differently. Our
research focuses on the areas of restitution and the costs of civil
proceedings, as we perceive these areas to have the greatest potential
for the Court's case law inconsistency.

Regarding the method, we employ a mix of quantitative methods (Bayesian
statistics) and legal computational studies. First, we narrow down the
selection of decisions to those falling under the given issues based on
the subject matter index. With access to data from the Beck-online
database, we can subsequently identify mutual citations in the
Constitutional Court's case law, including the positive/negative
relationship. The entire process of data wrangling is conducted using
our custom code in the R programming language with the Tidyverse
package. Meanwhile, we adapt Clark and Lauderdale's Bayesian model using
the available code to replicate their research. But we use the more
modern RStan as the Bayesian environment instead of RJags to simulate
the Markov Chains Monte Carlo: ``Stan's MCMC techniques are based on
Hamiltonian Monte Carlo, a more efficient and robust sampler than Gibbs
sampling or Metropolis Hastings for models with complex posteriors.''
(Brooks et al. 2011)

Subsequently, we analyse the results in connection with our research
question regarding the consistency of the Constitutional Court's case
law. Specifically, we compare the distributions of decision positions
within the given areas across the Court's senates and the plenum and
across different periods of the Constitutional Court.

\vspace{30pt}

\hypertarget{refs}{}
\begin{CSLReferences}{1}{0}
\leavevmode\vadjust pre{\hypertarget{ref-arnoldScalingCourtDecisions2023}{}}%
Arnold, Christian, Benjamin G. Engst, and Thomas Gschwend. 2023.
{``Scaling {Court Decisions} with {Citation Networks}.''} \emph{Journal
of Law and Courts} 11 (1): 25--44. \url{https://doi.org/10.1086/717420}.

\leavevmode\vadjust pre{\hypertarget{ref-brooksHandbookMarkovChain2011}{}}%
Brooks, Steve, Andrew Gelman, Galin Jones, and Xiao-Li Meng. 2011.
\emph{Handbook of {Markov Chain Monte Carlo}}. {CRC Press}.
\url{https://books.google.com?id=qfRsAIKZ4rIC}.

\leavevmode\vadjust pre{\hypertarget{ref-clarkLocatingSupremeCourt2010}{}}%
Clark, Tom S., and Benjamin Lauderdale. 2010. {``Locating {Supreme Court
Opinions} in {Doctrine Space}.''} \emph{American Journal of Political
Science} 54 (4): 871--90.
\url{https://doi.org/10.1111/j.1540-5907.2010.00470.x}.

\leavevmode\vadjust pre{\hypertarget{ref-fjelstulHowChamberSystem2021}{}}%
Fjelstul, Joshua. 2021. {``How the {Chamber System} at the {CJEU
Undermines} the {Consistency} of the {Court}'s {Application} of {EU
Law}.''} \emph{Journal of Law and Courts}, November, 717422.
\url{https://doi.org/10.1086/717422}.

\leavevmode\vadjust pre{\hypertarget{ref-martinDynamicIdealPoint2002}{}}%
Martin, Andrew D., and Kevin M. Quinn. 2002. {``Dynamic {Ideal Point
Estimation} via {Markov Chain Monte Carlo} for the {U}.{S}. {Supreme
Court}, 1953--1999.''} \emph{Political Analysis} 10 (2): 134--53.
\url{https://doi.org/10.1093/pan/10.2.134}.

\leavevmode\vadjust pre{\hypertarget{ref-segalIdeologicalValuesVotes1995}{}}%
Segal, Jeffrey A., Lee Epstein, Charles M. Cameron, and Harold J.
Spaeth. 1995. {``Ideological {Values} and the {Votes} of {U}.{S}.
{Supreme Court Justices Revisited}.''} \emph{The Journal of Politics} 57
(3): 812--23. \url{https://doi.org/10.2307/2960194}.

\end{CSLReferences}

\end{document}
