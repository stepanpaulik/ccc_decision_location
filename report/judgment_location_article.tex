% ARTICLE 2 ----
% This is just here so I know exactly what I'm looking at in Rstudio when messing with stuff.
% Options for packages loaded elsewhere
\PassOptionsToPackage{unicode}{hyperref}
\PassOptionsToPackage{hyphens}{url}
%
\documentclass[
  11pt,
]{article}
\usepackage{lmodern}
\usepackage{amssymb,amsmath}
\usepackage{ifxetex,ifluatex}
\ifnum 0\ifxetex 1\fi\ifluatex 1\fi=0 % if pdftex
  \usepackage[T1]{fontenc}
  \usepackage[utf8]{inputenc}
  \usepackage{textcomp} % provide euro and other symbols
\else % if luatex or xetex
  \usepackage{unicode-math}
  \defaultfontfeatures{Scale=MatchLowercase}
  \defaultfontfeatures[\rmfamily]{Ligatures=TeX,Scale=1}
\fi
% Use upquote if available, for straight quotes in verbatim environments
\IfFileExists{upquote.sty}{\usepackage{upquote}}{}
\IfFileExists{microtype.sty}{% use microtype if available
  \usepackage[]{microtype}
  \UseMicrotypeSet[protrusion]{basicmath} % disable protrusion for tt fonts
}{}
\usepackage{xcolor}
\IfFileExists{xurl.sty}{\usepackage{xurl}}{} % add URL line breaks if available
\urlstyle{same} % disable monospaced font for URLs
\usepackage[margin=1in]{geometry}
\setlength{\emergencystretch}{3em} % prevent overfull lines
\providecommand{\tightlist}{%
  \setlength{\itemsep}{0pt}\setlength{\parskip}{0pt}}
\setcounter{secnumdepth}{5}

\ifluatex
  \usepackage{selnolig}  % disable illegal ligatures
\fi
\newlength{\cslhangindent}
\setlength{\cslhangindent}{1.5em}
\newlength{\csllabelwidth}
\setlength{\csllabelwidth}{3em}
\newenvironment{CSLReferences}[2] % #1 hanging-ident, #2 entry spacing
 {% don't indent paragraphs
  \setlength{\parindent}{0pt}
  % turn on hanging indent if param 1 is 1
  \ifodd #1 \everypar{\setlength{\hangindent}{\cslhangindent}}\ignorespaces\fi
  % set entry spacing
  \ifnum #2 > 0
  \setlength{\parskip}{#2\baselineskip}
  \fi
 }%
 {}
\usepackage{calc}
\newcommand{\CSLBlock}[1]{#1\hfill\break}
\newcommand{\CSLLeftMargin}[1]{\parbox[t]{\csllabelwidth}{#1}}
\newcommand{\CSLRightInline}[1]{\parbox[t]{\linewidth - \csllabelwidth}{#1}\break}
\newcommand{\CSLIndent}[1]{\hspace{\cslhangindent}#1}


\title{Locating Czech Constitutional Court Decisions in Doctrine Space
and Measuring its Caselaw Consistency}
\author{true \and true}
\date{November 23, 2023}

% Jesus, okay, everything above this comment is default Pandoc LaTeX template. -----
% ----------------------------------------------------------------------------------
% I think I had assumed beamer and LaTex were somehow different templates.


\usepackage{kantlipsum}

\usepackage{abstract}
\renewcommand{\abstractname}{}    % clear the title
\renewcommand{\absnamepos}{empty} % originally center

\renewenvironment{abstract}
 {{%
    \setlength{\leftmargin}{0mm}
    \setlength{\rightmargin}{\leftmargin}%
  }%
  \relax}
 {\endlist}

\makeatletter
\def\@maketitle{%
  \newpage
%  \null
%  \vskip 2em%
%  \begin{center}%
  \let \footnote \thanks
      {\fontsize{18}{20}\selectfont\raggedright  \setlength{\parindent}{0pt} \@title \par}
    }
%\fi
\makeatother


\title{Locating Czech Constitutional Court Decisions in Doctrine Space
and Measuring its Caselaw Consistency }

\date{}

\usepackage{titlesec}

% 
\titleformat*{\section}{\large\bfseries}
\titleformat*{\subsection}{\normalsize\itshape} % \small\uppercase
\titleformat*{\subsubsection}{\normalsize\itshape}
\titleformat*{\paragraph}{\normalsize\itshape}
\titleformat*{\subparagraph}{\normalsize\itshape}

% add some other packages ----------

% \usepackage{multicol}
% This should regulate where figures float
% See: https://tex.stackexchange.com/questions/2275/keeping-tables-figures-close-to-where-they-are-mentioned
\usepackage[section]{placeins}



\makeatletter
\@ifpackageloaded{hyperref}{}{%
\ifxetex
  \PassOptionsToPackage{hyphens}{url}\usepackage[setpagesize=false, % page size defined by xetex
              unicode=false, % unicode breaks when used with xetex
              xetex]{hyperref}
\else
  \PassOptionsToPackage{hyphens}{url}\usepackage[draft,unicode=true]{hyperref}
\fi
}

\@ifpackageloaded{color}{
    \PassOptionsToPackage{usenames,dvipsnames}{color}
}{%
    \usepackage[usenames,dvipsnames]{color}
}
\makeatother
\hypersetup{breaklinks=true,
            bookmarks=true,
            pdfauthor={Štěpán Paulík (Humboldt Universität zu Berlin,
Institut für Sozialwissenschaften,
\href{mailto:stepan.paulik.1@hu-berlin.de}{\nolinkurl{stepan.paulik.1@hu-berlin.de}}) and Jaromír
Fronc (Charles University, Faculty of Law,
\href{mailto:jaromir.fronc@gmail.com}{\nolinkurl{jaromir.fronc@gmail.com}})},
             pdfkeywords = {empirical legal research, courts, dissents,
judicial behavior, political science, Bayesian statistics, regression
analysis},
            pdftitle={Locating Czech Constitutional Court Decisions in
Doctrine Space and Measuring its Caselaw Consistency},
            colorlinks=true,
            citecolor=blue,
            urlcolor=blue,
            linkcolor=magenta,
            pdfborder={0 0 0}}
\urlstyle{same}  % don't use monospace font for urls

% Add an option for endnotes. -----



% This will better treat References as a section when using natbib
% https://tex.stackexchange.com/questions/49962/bibliography-title-fontsize-problem-with-bibtex-and-the-natbib-package

% set default figure placement to htbp
\makeatletter
\def\fps@figure{htbp}
\makeatother



\usepackage{longtable}
\LTcapwidth=.95\textwidth
\linespread{1.05}
\usepackage{hyperref}
\usepackage{float}

\newtheorem{hypothesis}{Hypothesis}

\usepackage{setspace}

% trick for moving figures to back of document
% really wish we'd knock this shit off with moving tables/figures to back of document
% but, alas...

% 
% Optional code chunks ------
% SOURCE: https://stackoverflow.com/questions/50702942/does-rmarkdown-allow-captions-and-references-for-code-chunks



\begin{document}

% \textsf{\textbf{This is sans-serif bold text.}}
% \textbf{\textsf{This is bold sans-serif text.}}


% \maketitle

{% \usefont{T1}{pnc}{m}{n}
\setlength{\parindent}{0pt}
\thispagestyle{plain}
{%\fontsize{18}{20}\selectfont\raggedright
\maketitle  % title \par

}




{
   \vskip 13.5pt\relax \normalsize\fontsize{11}{12}
   \MakeUppercase{Štěpán Paulík}, \small{Humboldt Universität zu Berlin,
Institut für Sozialwissenschaften,
\href{mailto:stepan.paulik.1@hu-berlin.de}{\nolinkurl{stepan.paulik.1@hu-berlin.de}}}   \par \vskip -3.5pt \MakeUppercase{Jaromír
Fronc}, \small{Charles University, Faculty of Law,
\href{mailto:jaromir.fronc@gmail.com}{\nolinkurl{jaromir.fronc@gmail.com}}}   

}

}








\begin{abstract}

%    \hbox{\vrule height .2pt width 39.14pc}

    \vskip 8.5pt % \small

\noindent \small{Our research explores the estimation of positions of
Czech Constitutional Court decisions in a doctrine space using Bayesian
statistical model. Traditional methods of estimating ideological
positions suffer from limitations, prompting the adoption of new
text-as-data approaches empowered by advances in computational
technology and statistics. Two research teams have attempted to overcome
previous constraints and estimate judicial positions more accurately,
one in the SCOTUS context and one in the German lower courts context.
Our study implements the method of Clark and Lauderdale of estimating
the locations of SCOTUS decisions with positive or negative references
to its caselaw: the closer decisions are to each other, the more likely
they are to cite themselves positively and vice-versa. We combine our
own dataset of all CCC decisions with the data on citations provided to
us by Beck. We use the programming language R and the Bayesian engine
Stan to estimate the positions employing the Bayesian model of Clark and
Lauderdal. Estimating the positions allows us to examine the consistency
of the CC's case law across different senates and the plenum. We narrow
our analysis to areas of law in common doctrine space that are prone to
inconsistency, namely restitution cases and costs of civil proceedings.
The research contributes to harnessing the potential of machine learning
and quantitative methods in legal research and clarifies the factors
influencing caselaw consistency.}


\vskip 8.5pt \noindent \emph{Keywords}: empirical legal research,
courts, dissents, judicial behavior, political science, Bayesian
statistics, regression analysis \par

%    \hbox{\vrule height .2pt width 39.14pc}



\end{abstract}


\vskip -8.5pt

{
\hypersetup{linkcolor=black}
\setcounter{tocdepth}{2}
\tableofcontents
}

 % removetitleabstract

{
\setcounter{tocdepth}{2}
\tableofcontents
}

\setlength{\parindent}{16pt}
\setlength{\parskip}{0pt}

% We'll put doublespacing here
\doublespacing
% Remember to cut it out later before bib
The question of to what extent is court case law consistent has been of
interest of legal scholarship as well as legal practice for quite some
time. While the question has been posed in the context development of
caselaw over time, in the case of SCOTUS, or in the comparison between
differing courts, such as Germany's lower courts, we believe it to be
equally applicable to collegial courts organised into multiple chambers,
such as ECHR, ECJ or typically national constitutional courts. While
works about caselaw consistency are typical doctrinal, presenting
examples of mutually incompatible decisions, we study caselaw
consistency empirically employing quantitative methods.

\hypertarget{literature-review}{%
\section{Literature review}\label{literature-review}}

The issue of estimating caselaw consistency effectively boils down to
two issues: firstly, the estimation of location of a case in a
case-space and, secondly, to quantitatively conceptualizing consistency
of caselaw. In both aspects, we build on previous work - on the
empirical development of estimating location of a judicial decision in a
common case-space (Kornhauser 1992a, 1992b; Lax 2011) and theoretical
work into consistency law across court chambers (Fjelstul 2021).

\hypertarget{locating-court-decisions}{%
\subsection{Locating court decisions}\label{locating-court-decisions}}

Traditionally mainly American empirical legal science has long tried to
estimate a position of judges or judicial decisions, especially in the
context of political ideology of SCOTUS (Supreme Court of the United
States) decisions and judges. Based on this estimation, the research
sought to answer subsequent sociological research questions.
Spaeth-Segal (Segal et al. 1995) and Martin-Quinn (Martin and Quinn
2002) scores are among the examples of attempts to quantify the
ideological position of judgments or judges. However, these traditional
estimation methods suffer from various shortcomings, such as measuring
ideological positions indirectly (based on the party the decision was
made against) or using data not available outside the American context
(information about judges' voting) and, thus, making the research near
impossible to replicate elsewhere.

Recent developments in information technology enable more precise and
diverse work with data, specifically with text of court decisions as
data. This progress is starting to make its way into social and legal
sciences. Two research teams have utilised the progress to overcome the
aforementioned limitations in traditional estimation efforts of judicial
decisions and judges' positions.

Both models rest on the assumption that a location of a court decision
can be estimated based on its relationship to other court decisions and
sources of law. Courts typically justify their conclusion by reference
to other court decisions or laws that is typically close to them in a
case-space. Both teams make the following assumptions (a) each opinion
has a fixed location along a single latent dimensions delimited by a
common case-space and (b) they assume a proximity relationship, i.e.~the
probability of a (positive) reference to a court decision or other
sources of law increases the closer the decisions are to each other in a
case-space (Clark and Lauderdale 2010; Arnold, Engst, and Gschwend
2023).

The exact understanding of the relationship between a reference between
two decisions and their position on the latent dimension can be
understood in two manners. According to Clark and Lauderdale,
\emph{directional models} assume ``that an opinion is more likely to
cite a precedent if the precedent has the same {[}ideological{]}
polarity as the opinion'', whereas \emph{proximity models} assume that
``the probability of a positive citation fro man opinion to a precedent
decreases as the doctrine between the two opinions diverges.'' (Clark
and Lauderdale 2010, 875). Both studies, upon which we build, ultimately
base their model on the proximity model for it better reflects the
assumed practice of citing court decisions: ``Given the choice between
two precedents with identical doctrine and differing directional
outcomes, we believe that a justice in this situation would cite both
precedents positively.'' (Clark and Lauderdale 2010, 875) Moreover, the
proximity model better corresponds to the recent shift to model court
decisions in a case-space. Thus, we do not diverge from the original
studies. We now move on to the understanding and relevance of a
case-space.

\hypertarget{case-space-model}{%
\subsection{Case-space model}\label{case-space-model}}

The case-space model was developed in an attempt to model the
idiosyncrasies of court decision-making, i.e.~that a court is a body
resolving disputes, cases. The idea of cases must be taken seriously: we
must be able to represent a case and what rendering a judgment in any
case means.\\
The way to represent a case is by locating it in a space of of possible
cases, the case-space. A case then dennotes a ``legally relevant event
that has occured'' out of many that could've occured. A car in a 30km/h
zone could've in one case gone 28 km/h or in another 32 km/h. For the
sake of simplicity, the case-space is a uni-dimensional space, a line,
with the specific case being a scalar on that dimension.

To dispose of the case, the judge

\hypertarget{case-law-consistency}{%
\section{Case-law consistency}\label{case-law-consistency}}

\hypertarget{theoretical-conceptualization-and-operationalization-of-case-law-consistency}{%
\subsection{Theoretical conceptualization and operationalization of
case-law
consistency}\label{theoretical-conceptualization-and-operationalization-of-case-law-consistency}}

We now move on to elucidating how we understand the term
\emph{consistency} of case-law or of \emph{a court's application of the
law} (Fjelstul 2021, 2). Our understanding is firmly rooted in and
builts upon the definition provided by Fjelstul in his theoretical study
of the chamber system at the European Court of Justice (CJEU). Therein,
Fjelstul theoretically conceptualizes \emph{consistency} ``as the degree
to which the disposition of a case would change if it were decided by a
different chamber'' of a court.

Fjelstul operationalizes the theoretical definition as ``the variance of
the distribution of the expected probability that the plaintiff wins a
case across counterfactual chambers'' (Fjelstul 2021, 2)

In theory, both the theoretical definition and operationalization are
counterfactual as we can never observe how any given decision would be
decided by more than one actual court chamber. We can, however, attempt
to measure the trend across general population of cases falling into one
case-space. And while we are aware that comparison based on
observational data is not conclusive of any causal relationship, our
ambition is only to offer a descriptive analysis of the CCC's caselaw
without necessarily uncovering any causal relationship in the sense of
\emph{what type of underlying feature leads to higher/lower consistency
of CCC's caselaw?}

\hypertarget{literature-review-of-case-law-inconsistency-at-the-ccc}{%
\subsection{Literature review of case-law inconsistency at the
CCC}\label{literature-review-of-case-law-inconsistency-at-the-ccc}}

The objective of our contribution is to present research in which we
apply Clark and Lauderdale's method to the Czech Constitutional Court
since we believe their approach is more suitable for the institutional
setup of the Constitutional Court, where decisions refer primarily to
its own case law (Harašta et al. 2021, 220) and references to legal
provisions are not as diverse (Constitution, Charter of Fundamental
Rights and Freedoms). However, we also intend to try the German method
to compare the outcomes.

Based on the estimated positions of Constitutional Court decisions, we
will examine primarily the consistency of the Court's case law across
senates and the plenum. The interplay between the consistency of case
law and a court's internal structure has already been the subject of
quantitative empirical research in the context of the Court of Justice
of the European Union (Fjelstul 2021), which we build upon.

Scholarly literature is basically unanimous (Bobek and Kühn 2013) in the
notion that the case law of the Constitutional Court is inconsistent due
to the massive number of constitutional complaints and the conscious or
unconscious failure to unify the case law. The 3-member senates, into
which the 15-member Constitutional Court is divided, either are unaware
of conflicting case law, ignore it, or employ tricks to avoid its
binding nature. Moreover, it can be often difficult even to identify
that two decisions are in conflict with each other.

Literature also provides examples of areas where the Constitutional
Court's decision-making practice is particularly inconsistent. It can be
assumed that the level of inconsistency may be related to the
value-based or legal-political dimension of the issue, which causes
different judges to tend to decide differently. Our research focuses on
the areas of restitution matters and the costs of civil proceedings, as
we perceive these areas to have the greatest potential for the Court's
case law inconsistency.

The area of restitutions is widely considered to be one of the most
disputed legal-political issues after the Velvet Revolution in 1989.

\hypertarget{model-selection-and-specification}{%
\section{Model selection and
specification}\label{model-selection-and-specification}}

At first, we stood before a choice between two models developed for the
same purpose: estimation of a location of a court decision in a given
case-space. Both models share certain theoretical assumptions but
diverge on practical implementation.

Clark and Lauderdale examine SCOTUS, which, like the Czech
Constitutional Court (Ústavní soud - ÚS), primarily cites its own
decisions (Clark and Lauderdale 2010). They consider only these
citations, which they further divide into positive and negative based on
their relationship to the cited decision. More specifically, the authors
collected all freedom of religion and search and seizure cases and then
manually annotated citations among those decisions as either positive
and negative. The estimation was based on such an encapsulated,
self-referential dataset and on the citations being either positive or
negative.

In contrast, Arnold, Engst, and Gschwend study German general courts of
lower instances, their focus was not the highest
\emph{Bundesverfassungsgericht} but the consistency of caselaw on the
state level of the judiciary.

Ordinary lower courts typically refer to various legal provisions and
decisions of other courts (Arnold, Engst, and Gschwend 2023). Hence,
they only use information about references to other legal texts, which
can include not only decisions of the same court (as in the case of
SCOTUS) but also provisions of any legal regulation or decisions of any
court.

The feature selection of the two models is justified by the nature of
typical citations on the highest level of the judiciary - SCOTUS or
constitutional courts - and on the lower levels of ordinary judiciary.
The caselaw citations of SCOTUS or constitutional courts are typically
self-referential. Citations of laws typically refer to vague
constitutional provisions, which will be typically shared by cases on
both extremes on a latent dimensions, and as such do not contain any
information as to the location of the decision on the latent dimensions.
In contrast, a lower court decision will typically cite decisions of
higher courts and very precisely formulated legal provisions. In
practice, a conservative freedom of speech decision will refer to the
same vague and broadly-formulated provisions as a progressive freedom of
speech decision, whereas strict lower court defamation decisions will
altogether refer to different legal provisions than lenient defamation
decisions.

\hypertarget{method}{%
\section{Method}\label{method}}

Regarding the method, we employ a mix of quantitative methods (Bayesian
statistics) and legal computational studies. First, we narrow down the
selection of decisions to those falling under the given issues based on
the subject matter index. With access to data from the Beck-online
database, we can subsequently identify mutual citations in the
Constitutional Court's case law, including the positive/negative
relationship. The entire process of data wrangling is conducted using
our custom code in the R programming language with the Tidyverse
package. Meanwhile, we adapt Clark and Lauderdale's Bayesian model using
the available code to replicate their research. But we use the more
modern RStan as the Bayesian environment instead of RJags to simulate
the Markov Chains Monte Carlo: ``Stan's MCMC techniques are based on
Hamiltonian Monte Carlo, a more efficient and robust sampler than Gibbs
sampling or Metropolis Hastings for models with complex posteriors.''
(Brooks et al. 2011)

\hypertarget{refs}{}
\begin{CSLReferences}{1}{0}
\leavevmode\vadjust pre{\hypertarget{ref-arnoldScalingCourtDecisions2023}{}}%
Arnold, Christian, Benjamin G. Engst, and Thomas Gschwend. 2023.
{``Scaling {Court Decisions} with {Citation Networks}.''} \emph{Journal
of Law and Courts} 11 (1): 25--44. \url{https://doi.org/10.1086/717420}.

\leavevmode\vadjust pre{\hypertarget{ref-bobekJudikaturaPravniArgumentace2013}{}}%
Bobek, Michal, and Zdeněk Kühn, eds. 2013. \emph{Judikatura a právní
argumentace}. 2. vydání. {Praha}: {Auditorium}.

\leavevmode\vadjust pre{\hypertarget{ref-brooksHandbookMarkovChain2011}{}}%
Brooks, Steve, Andrew Gelman, Galin Jones, and Xiao-Li Meng. 2011.
\emph{Handbook of {Markov Chain Monte Carlo}}. {CRC Press}.
\url{https://books.google.com?id=qfRsAIKZ4rIC}.

\leavevmode\vadjust pre{\hypertarget{ref-clarkLocatingSupremeCourt2010}{}}%
Clark, Tom S., and Benjamin Lauderdale. 2010. {``Locating {Supreme Court
Opinions} in {Doctrine Space}.''} \emph{American Journal of Political
Science} 54 (4): 871--90.
\url{https://doi.org/10.1111/j.1540-5907.2010.00470.x}.

\leavevmode\vadjust pre{\hypertarget{ref-fjelstulHowChamberSystem2021}{}}%
Fjelstul, Joshua. 2021. {``How the {Chamber System} at the {CJEU
Undermines} the {Consistency} of the {Court}'s {Application} of {EU
Law}.''} \emph{Journal of Law and Courts}, November, 717422.
\url{https://doi.org/10.1086/717422}.

\leavevmode\vadjust pre{\hypertarget{ref-harastaCitacniAnalyzaJudikatury2021}{}}%
Harašta, Jakub, Terezie Smejkalová, Jaromír Šavelka, and Radim Polčák.
2021. \emph{Citační analýza judikatury}. Vydání první. Právní
monografie. {Praha}: {Wolters Kluwer}.

\leavevmode\vadjust pre{\hypertarget{ref-kornhauserModelingCollegialCourts1992a}{}}%
Kornhauser, Lewis A. 1992a. {``Modeling {Collegial Courts}. {II}. {Legal
Doctrine}.''} \emph{Journal of Law, Economics and Organization} 8: 441.
\url{https://heinonline.org/HOL/Page?handle=hein.journals/jleo8&id=449&div=&collection=}.

\leavevmode\vadjust pre{\hypertarget{ref-kornhauserModelingCollegialCourts1992}{}}%
---------. 1992b. {``Modeling Collegial Courts {I}:
{Path-dependence}.''} \emph{International Review of Law and Economics}
12 (2): 169--85. \url{https://doi.org/10.1016/0144-8188(92)90034-O}.

\leavevmode\vadjust pre{\hypertarget{ref-laxNewJudicialPolitics2011}{}}%
Lax, Jeffrey R. 2011. {``The {New Judicial Politics} of {Legal
Doctrine}.''} \emph{Annual Review of Political Science} 14 (1): 131--57.
\url{https://doi.org/10.1146/annurev.polisci.042108.134842}.

\leavevmode\vadjust pre{\hypertarget{ref-martinDynamicIdealPoint2002}{}}%
Martin, Andrew D., and Kevin M. Quinn. 2002. {``Dynamic {Ideal Point
Estimation} via {Markov Chain Monte Carlo} for the {U}.{S}. {Supreme
Court}, 1953--1999.''} \emph{Political Analysis} 10 (2): 134--53.
\url{https://doi.org/10.1093/pan/10.2.134}.

\leavevmode\vadjust pre{\hypertarget{ref-segalIdeologicalValuesVotes1995}{}}%
Segal, Jeffrey A., Lee Epstein, Charles M. Cameron, and Harold J.
Spaeth. 1995. {``Ideological {Values} and the {Votes} of {U}.{S}.
{Supreme Court Justices Revisited}.''} \emph{The Journal of Politics} 57
(3): 812--23. \url{https://doi.org/10.2307/2960194}.

\end{CSLReferences}

\end{document}
