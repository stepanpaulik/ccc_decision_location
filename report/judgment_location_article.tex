% Options for packages loaded elsewhere
\PassOptionsToPackage{unicode}{hyperref}
\PassOptionsToPackage{hyphens}{url}
%
\documentclass[
]{article}
\usepackage{amsmath,amssymb}
\usepackage{iftex}
\ifPDFTeX
  \usepackage[T1]{fontenc}
  \usepackage[utf8]{inputenc}
  \usepackage{textcomp} % provide euro and other symbols
\else % if luatex or xetex
  \usepackage{unicode-math} % this also loads fontspec
  \defaultfontfeatures{Scale=MatchLowercase}
  \defaultfontfeatures[\rmfamily]{Ligatures=TeX,Scale=1}
\fi
\usepackage{lmodern}
\ifPDFTeX\else
  % xetex/luatex font selection
\fi
% Use upquote if available, for straight quotes in verbatim environments
\IfFileExists{upquote.sty}{\usepackage{upquote}}{}
\IfFileExists{microtype.sty}{% use microtype if available
  \usepackage[]{microtype}
  \UseMicrotypeSet[protrusion]{basicmath} % disable protrusion for tt fonts
}{}
\makeatletter
\@ifundefined{KOMAClassName}{% if non-KOMA class
  \IfFileExists{parskip.sty}{%
    \usepackage{parskip}
  }{% else
    \setlength{\parindent}{0pt}
    \setlength{\parskip}{6pt plus 2pt minus 1pt}}
}{% if KOMA class
  \KOMAoptions{parskip=half}}
\makeatother
\usepackage{xcolor}
\usepackage[margin=1in]{geometry}
\usepackage{graphicx}
\makeatletter
\def\maxwidth{\ifdim\Gin@nat@width>\linewidth\linewidth\else\Gin@nat@width\fi}
\def\maxheight{\ifdim\Gin@nat@height>\textheight\textheight\else\Gin@nat@height\fi}
\makeatother
% Scale images if necessary, so that they will not overflow the page
% margins by default, and it is still possible to overwrite the defaults
% using explicit options in \includegraphics[width, height, ...]{}
\setkeys{Gin}{width=\maxwidth,height=\maxheight,keepaspectratio}
% Set default figure placement to htbp
\makeatletter
\def\fps@figure{htbp}
\makeatother
\setlength{\emergencystretch}{3em} % prevent overfull lines
\providecommand{\tightlist}{%
  \setlength{\itemsep}{0pt}\setlength{\parskip}{0pt}}
\setcounter{secnumdepth}{-\maxdimen} % remove section numbering
\ifLuaTeX
  \usepackage{selnolig}  % disable illegal ligatures
\fi
\IfFileExists{bookmark.sty}{\usepackage{bookmark}}{\usepackage{hyperref}}
\IfFileExists{xurl.sty}{\usepackage{xurl}}{} % add URL line breaks if available
\urlstyle{same}
\hypersetup{
  hidelinks,
  pdfcreator={LaTeX via pandoc}}

\author{}
\date{\vspace{-2.5em}}

\begin{document}

\begin{center}\rule{0.5\linewidth}{0.5pt}\end{center}

output: stevetemplates::article2: keep\_tex: true biblio-style: apsr
title: ``Locating Czech Constitutional Court Decisions in Doctrine Space
and Measuring its Caselaw Consistency'' author: - name: Štěpán Paulík
affiliation: Humboldt Universität zu Berlin, Institut für
Sozialwissenschaften,
\href{mailto:stepan.paulik.1@hu-berlin.de}{\nolinkurl{stepan.paulik.1@hu-berlin.de}}
- name: Jaromír Fronc affiliation: Charles University, Faculty of Law,
\href{mailto:jaromir.fronc@gmail.com}{\nolinkurl{jaromir.fronc@gmail.com}}
abstract: ``Our research explores the estimation of positions of Czech
Constitutional Court decisions in a doctrine space using Bayesian
statistical model. Traditional methods of estimating ideological
positions suffer from limitations, prompting the adoption of new
text-as-data approaches empowered by advances in computational
technology and statistics. Two research teams have attempted to overcome
previous constraints and estimate judicial positions more accurately,
one in the SCOTUS context and one in the German lower courts context.
Our study implements the method of Clark and Lauderdale of estimating
the locations of SCOTUS decisions with positive or negative references
to its caselaw: the closer decisions are to each other, the more likely
they are to cite themselves positively and vice-versa. We combine our
own dataset of all CCC decisions with the data on citations provided to
us by Beck. We use the programming language R and the Bayesian engine
Stan to estimate the positions employing the Bayesian model of Clark and
Lauderdal. Estimating the positions allows us to examine the consistency
of the CC's case law across different senates and the plenum. We narrow
our analysis to areas of law in common doctrine space that are prone to
inconsistency, namely restitution cases and costs of civil proceedings.
The research contributes to harnessing the potential of machine learning
and quantitative methods in legal research and clarifies the factors
influencing caselaw consistency.'' date: ``October 24, 2023'' geometry:
margin=1in fontsize: 11pt endnote: no sansitup: FALSE header-includes: -

\usepackage{longtable}

\begin{itemize}
\tightlist
\item
  \LTcapwidth=.95\textwidth

  \begin{itemize}
  \item
    \linespread{1.05}
  \item
    \usepackage{hyperref}

    bibliography: ``bibliography.bib''
  \end{itemize}
\end{itemize}

The question of to what extent is court case law consistent has been of
interest of legal scholarship as well as legal practice for quite some
time. While the question has been posed in the context development of
caselaw over time, in the case of SCOTUS, or in the comparison between
differing courts, such as Germany's lower courts, we believe it to be
equally applicable to collegial courts organised into multiple chambers,
such as ECHR, ECJ or typically national constitutional courts.

While works about caselaw consistency are typical doctrinal, we study
caselaw consistency empirically.

\hypertarget{method}{%
\section{Method}\label{method}}

We believe the issue of caselaw consistency to boil down to two issues:
firstly, the estimation of location of a case in a case-space and,
secondly, to conceptualizing consistency of caselaw as a variation of
thereof among different units of interest, court chambers in our
context.

Especially American empirical legal science has long been trying to
estimate the ideological position of judges or judicial decisions,
especially in the context of SCOTUS (Supreme Court of the United
States). Based on this estimation, the research seeks to answer
subsequent sociological research questions. Spaeth-Segal
{[}@segalIdeologicalValuesVotes1995{]} and Martin-Quinn
{[}@martinDynamicIdealPoint2002{]} scores are among the examples of
attempts to quantify the ideological position of judges. However, these
traditional estimation methods suffer from various shortcomings, such as
measuring ideological positions indirectly (based on the party the
decision was made against) or using data not available outside the
American context (information about judges' voting) and, thus, making
the research near impossible to replicate elsewhere.

Recent developments in information technology enable more precise and
diverse work with data, specifically with text as data. This progress is
starting to make its way into social and legal sciences. Two research
teams have utilised the progress to overcome the aforementioned
limitations in traditional estimation efforts of judicial decisions and
judges' positions. Both teams operate on the premise that the ``closer''
judicial decisions or other legal documents are to each other, the
greater the likelihood that they will refer to one another. They employ
similar statistical models to estimate the position of a given decision
based on this premise. Moreover, their methods share the use of Bayesian
statistics, whose expansion has been facilitated by the development of
computational techniques. Despite these similarities, there is a
fundamental difference between the two research teams in determining
which intertextual reference is relevant for identifying the mutual
position of judicial decisions.

Clark and Lauderdale examine SCOTUS, which, like the Czech
Constitutional Court (Ústavní soud - ÚS), primarily cites its own
decisions {[}@clarkLocatingSupremeCourt2010{]}. Therefore, they consider
only these citations, which they further divide into positive and
negative based on their relationship to the cited decision. In contrast,
Arnold, Engst, and Gschwend study German general courts of lower
instances, which primarily refer to various legal provisions and
decisions of other courts (mostly higher in the judicial pyramide)
{[}@arnoldScalingCourtDecisions2023{]}. Hence, they only use information
about references to other legal texts, which can include not only
decisions of the same court (as in the case of SCOTUS) but also
provisions of any legal regulation or decisions of any court. Despite
these differences, the outcome of both efforts is similar: estimating
the placement of judicial decisions on a common dimension, whose
substantive meaning must be subsequently interpreted by researchers.

The objective of our contribution is to present research in which we
apply Clark and Lauderdale's method to the Czech Constitutional Court
since we believe their approach is more suitable for the institutional
setup of the Constitutional Court, where decisions refer primarily to
its own case law, and references to legal provisions are not as diverse
(Constitution, Charter of Fundamental Rights and Freedoms). However, we
also intend to try the German method to compare the outcomes.

Based on the estimated positions of Constitutional Court decisions, we
will examine primarily the consistency of the Court's case law across
senates and the plenum. The interplay between the consistency of case
law and the Court's internal structure has already been the subject of
quantitative empirical research {[}@fjelstulHowChamberSystem2021{]},
which we build upon.

The case law of the Constitutional Court is inconsistent due to the
massive number of constitutional complaints and the conscious or
unconscious failure to unify the case law. The 3-member senates, into
which the 15-member Constitutional Court is divided, either are unaware
of conflicting case law, ignore it, or employ tricks to avoid its
binding nature. Moreover, it can be often difficult even to identify
that two decisions are in conflict with each other.

Scholarly literature provides examples of areas where the Constitutional
Court's decision-making practice is particularly inconsistent (e.g.,
restitution matters). It can be assumed that the level of inconsistency
may be related to the value-based or legal-political dimension of the
issue, which causes different judges to tend to decide differently. Our
research focuses on the areas of restitution and the costs of civil
proceedings, as we perceive these areas to have the greatest potential
for the Court's case law inconsistency.

Regarding the method, we employ a mix of quantitative methods (Bayesian
statistics) and legal computational studies. First, we narrow down the
selection of decisions to those falling under the given issues based on
the subject matter index. With access to data from the Beck-online
database, we can subsequently identify mutual citations in the
Constitutional Court's case law, including the positive/negative
relationship. The entire process of data wrangling is conducted using
our custom code in the R programming language with the Tidyverse
package. Meanwhile, we adapt Clark and Lauderdale's Bayesian model using
the available code to replicate their research. But we use the more
modern RStan as the Bayesian environment instead of RJags to simulate
the Markov Chains Monte Carlo: ``Stan's MCMC techniques are based on
Hamiltonian Monte Carlo, a more efficient and robust sampler than Gibbs
sampling or Metropolis Hastings for models with complex posteriors.''
{[}@brooksHandbookMarkovChain2011{]}

Subsequently, we analyse the results in connection with our research
question regarding the consistency of the Constitutional Court's case
law. Specifically, we compare the distributions of decision positions
within the given areas across the Court's senates and the plenum and
across different periods of the Constitutional Court.

\end{document}
